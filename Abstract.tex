
\indent There is ongoing research at the Federal Aviation Administration (FAA) and
other private industries to examine a concept for delegated separation in multiple classes
of airspace to allow unmanned aircraft systems (UAS) to remain well clear of other
aircraft. Detect and Avoid (DAA) capabilities are one potential technology being
examined to maintain separation. To evaluate these DAA capabilities, input traffic
scenarios are simulated based on either simple geometric aircraft trajectories or recorded
traffic scenarios and are replayed in a simulator. However, these approaches are limited
by the breadth of the traffic recordings available. This thesis derives a new mathematical
algorithm that uses great circle navigation equations in an Earth spherical model and an
accurate aircraft performance model to generate realistic aircraft encounters in any
airspace. This algorithm is implemented in a program called Encounters from Actual
Trajectories (EnAcT) and uses several user inputs defining the encounter events, called
encounter properties. Given these encounter properties, the program generates two 4-
dimensional flight trajectories that satisfy these properties. This thesis also describes a
study performed to determine the appropriate encounter properties to use for developing
the encounters. This encounter generator could be used to evaluate DAA systems as well
as initiate research in automation for encounter detection and resolution.

