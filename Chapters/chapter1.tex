
\chapter{Introduction}

The Federal Aviation Administration (FAA) and private industry collectively within the Radio Technical Commission for Aeronautics (RTCA) Special Committee (SC) 228 are working to define the minimum operational performance standards (MOPS) for unmanned aircraft systems (UAS) to allow for safe integration of UAS into the National Airspace System (NAS). Part of defining the MOPS is ensuring the Detect and Avoid Algorithms (DAA) on UAS function properly to adhere to the See and Avoid requirement of the 14 CFR 91 - General Operating and Flight Rules imposed by the FAA. The FAA Federal Aviation Regulation (FAR) requirement of “See and Avoid” is essential to safe flight in manned aircraft. See and Avoid refers to the ability of the pilot to visually scan the surrounding airspace and determine possible risks. It is then the pilot's responsibility to avoid these risks by following the rules defined in the FAR or by any means in the case of an emergency. Thus, UASs without a pilot on board the aircraft are required to carry DAA systems that comply with 14 CFR 91.

The DAA system provides preventive, corrective, and warning guidance to the UAS pilot to assist in preventing loss of separation with other aircraft. These systems have significant technical challenges in establishing and maintaining the relative position of one or more external threats (i.e., aircraft) during an encounter. The entire DAA system consists of surveillance sensors used to detect an intruder, tracker to fuse and filter multiple sources of information to provide one single track to the DAA algorithm.

To ensure DAA systems can detect other aircraft, the systems need to be tested. Evaluating these systems requires scenarios in which two or more aircraft have a close encounter with each other. An encounter event is when two aircraft of interest are within a defined range of each other, typically close enough to cause a concern with air traffic control. A subset of encounters, called conflicts, are legally defined as a loss of separation between the two aircraft. Encounter/Conflict definitions change with the different types of airspaces in the NAS. In EnRoute (Class A) airspace, separation distances are usually greater than those in terminal (Class B, C, D, E) airspaces. The DAA system must be tested with cases of encounters and conflicts in all airspace types to ensure they can correctly identify when the UAS may stray too close to another aircraft. This work focuses on encounters with two aircraft.

In the NAS, no conflicts exist since the controllers keep the traffic separated thus preventing the use of unmodified recorded air traffic. Scenarios need to be generated to be able to simulate these desired encounters. Current capabilities of generating rely on simulated input traffic scenarios based on either simple linear aircraft trajectories or recorded traffic scenarios \cite{oaks:2001, paglione:2003, oaks:2002a, oaks:2002b, ritchie:2016}. However, these approaches are either limited by the simplicity of their models or by the breadth of the traffic recordings available. Thus, there is a need for more realistic and complex algorithms to generate a large set of user specified traffic scenarios.

In the literature, the generation of conflicting aircraft trajectories can be divided into three categories: (1) fitting probabilistic distributions and statistical models over the range of encounter variables \cite{kochenderfer-correlated:2008, kochenderfer-uncorrelated:2008}; (2) using recorded air-traffic data through time-shifting the flights \cite{oaks:2001, paglione:2003, oaks:2002a, oaks:2002b, ritchie:2016}; and (3) ad-hoc generation of aircraft trajectories for the main purpose of studying conflict detection \cite{malaek:2011, ming:2011, meng:2012, liu:2014, yang:2017}. Each of these approaches considers different assumptions and addresses different concerns regarding what type of conflicts are generated and why they are needed.

The encounter model in \cite{kochenderfer-correlated:2008}, developed by the Massachusetts Institute of Technology's (MIT) Lincoln Laboratory, describes a probabilistic aircraft encounter model. This model is currently limited to generating encounters in the EnRoute environment, where flights are typically cruising and do not change altitude often. Also, the encounters are all generated at random (according to the probabilistic properties), and you cannot specify what kind of encounter you would like to have. MIT Lincoln Laboratory is currently working on a new encounter model, but it is not ready for generating encounters at the time of this publication.

The FAA has traditionally generated aircraft conflicts and encounters through the use of time-shifting the flights in a recorded air traffic scenario \cite{oaks:2001, paglione:2003}. These algorithms consider the recorded flight data of aircraft that have flown in the NAS and shift the position of the aircraft in time to induce encounters. The resulting trajectories contain the same physical flight position data as the original recorded traffic data, but the aircraft fly these tracks at new times. A genetic algorithm was implemented to determine the optimal time-shift values for each flight in the scenario \cite{oaks:2002a}. Even though more encounter criteria have been considered \cite{oaks:2002b} and the algorithm has been improved since its original version \cite{ritchie:2016}, current algorithms only generate encounters from flights that have existed in the NAS. This prevents the user from specifying the exact parameters for each conflict, which is useful for testing specific conditions. These algorithms have only been tested on EnRoute (Class A) airspace, and not on data from the terminal environment. These methods would need to be evaluated for use in terminal environments and have had limited use beyond EnRoute airspace.

Other studies did not focus on the problem of aircraft conflict generation, but considered the issue of conflict detection or resolution \cite{malaek:2011, ming:2011, liu:2014, yang:2017}. Aircraft conflicts in these studies are typically fixed and have only examined a subset of conflict types. The trajectories of these aircraft are typically straight, and use coordinates based on a flat-Earth system. While the encounter properties are typically specified, the method in which they are created is not conducive to generating millions of encounters in a reasonable time span. This paper addresses the need to simulate encounter traffic events for the aim of testing the DAA capabilities of UAS in all airspace types. 

In this paper, we propose an algorithm to calculate the bearing between two aircraft given a defined encounter scenario. Specifically, we propose a new algorithm that uses a spherical Earth model to retain accuracy in the calculation of the bearing between the aircraft in a defined encounter scenario. A program called Encounters from Actual Trajectories (EnAcT) was written to use this algorithm to generate encounters using the derived algorithm and the defined encounter properties. The output of the program will cover gaps that exist in the types of encounters that are needed for testing, including the use of performance parameters from EuroControl's Base of Aircraft Data (BADA) \cite{bada:v3} to generate pairs of 4D trajectories that satisfy a specified set of encounter event properties. 

To determine the encounter properties that would be realistic to what is found in the NAS, a study was performed during this work to determine these values. Statistical distributions were created from real-world data for each encounter property and used by the program that was developed to generate numerous encounters that would be realistic. This allows the FAA to test UAS DAAs based on what is more likely to be found in actual air traffic operations.
