\chapter{Conclusion and Future Work}
In this thesis, a new mathematical algorithm is proposed that uses great circle navigation equations in an Earth spherical model and an accurate aircraft performance model to generate realistic aircraft encounters in any airspace. The algorithm is implemented in a program called Encounters from Actual Trajectories (EnAcT). Several user inputs defining the encounter events, called encounter properties, were derived via a study on trajectories found within the National Airspace System (NAS) in the Federal Aviation Administration (FAA) controlled airspace. Given these encounter properties, the program is proven to generate accurate, two 4-dimensional flight trajectories that satisfy these properties. This encounter generator was given to the FAA for use in generating more accurate trajectories for testing Detect-and-Avoid (DAA) algorithms.

If this work were to continue in the future, more performance models could be incorporated to include more Unmanned Aircraft Systems (UAS), as EUROCONTROL’s BADA 3 doesn’t include many variants of UAS. Also, the algorithm assumes that the aircraft are not turning during the encounter event, and this could be investigated to improve upon the accuracy of the encounters generated. For the conflict properties study, more recorded traffic will need to be examined to determine the empirical distributions for the NAS conflict properties throughout the year. Also, conflicts within terminal airspace could be studied and used as input to the algorithm. This will allow more testing for UAS in these areas with more realistic trajectories. Lastly, with advancements in generative artificial intelligence (AI), these trajectories could be generated via an AI algorithm in the future, incorporating the entire NAS as a whole.	