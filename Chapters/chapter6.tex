\chapter{Results of Encounter Study}
This chapter describes results from one of the ARTCCs examined in the study. The selected ARTCC for this section is Denver center (ZDV). This center had 323 hypothetical encounter events generated by the Trajectory Conflict Probe software. 

\section{Empirical Distributions}
Minimum separation between two aircraft is typically 5 NM horizontally and 1,000 ft. (or 2,000 ft. if the aircraft is not RVSM capable) vertically when in Class A (en route) airspace. When examining the predicted encounter events produced by Trajectory Conflict Probe, the distance between the aircraft is used to categorize the flights into bins. For horizontal separation, there is a bin for every NM between 0 and 5. Each bin’s lower bound is inclusive, while the upper bound is exclusive. \ref{fig:horzsepdistchart} presents an example of the binned predicted encounter events for ZDV. In this example, the number of flights in the first four bins is similar, and the number of flights in the 4 to 5 NM bin is lower.

\begin{figure}[H]
\centering
\includegraphics{Figures/HorizontalSeparationDistanceChart.png}
\caption{Total Number of Hypothetical Encounter Alerts for each Horizontal Separation Bin in ZDV}
\label{fig:horzsepdistchart}
\end{figure}
~\\

The vertical separation between the aircraft is binned in 500-foot intervals, from 0 to 2000 ft. As with the horizontal separation bins, the bins for vertical separation have an inclusive lower bound and an exclusive upper bound. \ref{fig:vertsepdistchart} displays the results from examining ZDV’s vertical separation between aircraft during encounter events.

\begin{figure}[H]
\centering
\includegraphics{Figures/VerticalSeparationDistanceChart.png}
\caption{Total Number of Hypothetical Encounter Alerts for each Vertical Separation Bin in ZDV}
\label{fig:vertsepdistchart}
\end{figure}
~\\

In ZDV, most of the predicted encounter events occurred between 0 and 500 ft. Some airspaces had different distributions, including near uniform, while others were similar to ZDV.

Encounter angle, or the difference in heading between aircraft relative to the subject aircraft, is split into three different categories: in-trail/overtake, crossing (on the left and on the right), and head-on. Crossing is split into two bins: 15-90° (on the left) and 90-165° (on the right), while the other two categories represent one bin each. This gives four bins for encounter angle. \ref{fig:encanglechart} illustrates the distribution of encounter angle between aircraft in ZDV’s encounter events.

\begin{figure}[H]
\centering
\includegraphics{Figures/EncounterAngleDistributionChart.png}
\caption{Total Number of Hypothetical Encounter Alerts for each Encounter Angle Bin in ZDV}
\label{fig:encanglechart}
\end{figure}
~\\

For this ARTCC, most of the predicted events happened in crossing, while very few head-on encounter events were predicted.

The vertical phase of flight type is the combination of the vertical phase of flight of each aircraft in an encounter event. Each encounter event is placed into a bin that corresponds to the vertical phase of flight of the pair. \ref{fig:vpofchart} presents the results of examining this property in ZDV. In ZDV, most of the flights are level during an encounter event, while some occurred while one aircraft was ascending or descending and the other was level.

\begin{figure}[H]
\centering
\includegraphics{Figures/VerticalPoFChart.png}
\caption{Total Number of Vertical Phases of Flight Pairs for Aircraft in Hypothetical Encounter Alerts in ZDV}
\label{fig:vpofchart}
\end{figure}
~\\

The aircraft type parameter is simply the number of each aircraft type that was involved in the predicted encounter event that was generated by the Trajectory Conflict Probe software. Both aircraft involved in the event are counted toward the total for each aircraft type. \ref{fig:aircraftchart} describes the number of each aircraft type present in the predicted events for ZDV.

\begin{figure}[H]
\centering
\includegraphics[scale=0.95]{Figures/AircraftTypeDistributionChart.png}
\caption{Total Number of Aircraft Types Involved in Hypothetical Encounter Alerts in ZDV}
\label{fig:aircraftchart}
\end{figure}

ZDV had 66 different aircraft types involved in encounter events, with at least two aircraft in each bin. The frequencies in this graph mirror the representation of aircraft types found in the airspace. 

/section{Encounter Event Location}
The location of an encounter event is described by three parameters: latitude, longitude, and altitude. A distribution is fit to the altitude data using JMP\textregistered analytical software, producing the output shown in \ref{fig:altitudechart} for ZDV.

~\\
\begin{figure}[H]
\centering
\includegraphics[scale=0.95]{Figures/AltitudeDistributionChart.png}
\caption{Total Number of Hypothetical Encounter Alerts for each Altitude Bin in ZDV}
\label{fig:altitudechart}
\end{figure}
~\\

A three normal mixture model is the best fit for this distribution of encounter event altitudes in ZDV. The other ARTCCs’ altitudes also are modeled using a three normal mixture model.

To characterize the distribution of horizontal location of encounters, K-means clustering is used with 20 clusters. \ref{fig:kmeanschart} shows a diagram of the clusters produced from ZDV.

\begin{figure}[H]
\centering
\includegraphics{Figures/KMeansClusterChart_HorizontalPosition.png}
\caption{K-means Cluster Diagram of ZDV}
\label{fig:kmeanschart}
\end{figure}
~\\

Each colored circle in \ref{fig:kmeanschart} forms a cluster produced by the K-means algorithm. The solid dots are the locations of predicted encounter events that the K-means algorithm uses to generate these clusters. Latitude coordinates are on the ordinate while the longitude coordinates are on the abscissa. Each cluster is modeled by a two-dimensional normal distribution with point estimates for the mean latitude and longitude dimensions and associated standard deviations. These models, in combination with the altitude distribution, estimated separately, represent the distribution of the location of encounter events throughout ZDV. 